\pdfminorversion=4
\documentclass[]{article}

\usepackage{amssymb,latexsym,geometry,xcolor,framed,url,standalone}     % Standard packages
\usepackage[fleqn]{amsmath}
\usepackage{colors}

\newgeometry{left=1cm,top=1cm,right=1cm,bottom=2cm}

\newcommand{\red}[1]{\color{red}\boldsymbol{#1}\color{black}}

\begin{document}
\setbool{@fleqn}{false}
\allowdisplaybreaks
\raggedbottom

\section*{Euler Project - Problem 6 - Sum Square Difference}
\subsection*{Task and corollaries}
\vspace{10pt}
\begin{graybox}
The sum of the squares of the first ten natural numbers is,
\[1^2 + 2^2 + \ldots + 10^2 = 385.\]
The square of the sum of the first ten natural numbers is,
\[(1 + 2 + \ldots + 10)^2 = 55^2 = 3025.\]

Hence the difference between the sum of the squares of the first ten natural numbers and the square of the sum is $3025 - 385 = 2640$.\\~\\
Find the difference between the sum of the squares of the first one hundred natural numbers and the square of the sum.
\begin{flushleft}
{\color{gray}\textit{Source: \url{https://projecteuler.net/problem=6}}}
\end{flushleft}
\end{graybox}

\vspace{10pt}
\noindent The needed corollaries for this proof can be studied in Appendix A.\\Assumed is knowledge of summation laws and arithmetic series.

\subsection*{Finding the pattern}
To better describe the patterns, let's assign letters to the specific natural numbers, starting with $a=1$, $b=2$, and so on.\\

Demonstrated on $N=2$:
\begin{align*}
\mathbf{N} &= 2\\
1^2 + 2^2 &\rightarrow \red{a^2 + b^2}  \\
&= 5\\
(1 + 2)^2 &\rightarrow (a + b)^2 = \red{a^2}  + 2ab + \red{b^2}  \\
&= 9\\[2ex]
\Delta &= 4\\
%
\intertext{Notice the highlighted terms appear in both expressions. As a consequence of the Binomial expansion the first and last terms will always appear squared. Clearly, the difference consists purely of the other terms in the expansion.\newline\newline\indent Let's shift our focus on the expansion of the second expression (sum squared) only:}
%
\mathbf{N} &= 3 \\
(a + b + c)^2 &= (a^2 + ab + ac) + (ba + b^2 + bc) + (ca + cb + c^2) \\
&= a^2 + b^2 + c^2 + \red{2ab + 2ac + 2bc} \\[3ex]
\mathbf{N} &= 4 \\
(a + b + c + d)^2 &= ... + \red{2ab + 2ac + 2ad + 2bc + 2bd + 2cd}  \\
%
\intertext{The polynomial expansion gives a hint as to what's going on - after summing the terms, we may notice that for any $N$, the difference is the product of each two terms (i.e. a, b, ...) multiplied by two.}
\end{align*}
\newpage
\begin{align*}
\intertext{To see the pattern even clearer, with the ever-increasing amount of terms, we may start with $a$, and letter by letter factor them out, as such:}
\mathbf{N} &= 2\\
\Delta &= 2~[a(b)]\\[3ex]
\mathbf{N} &= 3\\
\Delta &= 2~[a(b+c) + b(c)]\\[3ex]
\mathbf{N} &= 4\\
\Delta &= 2~[a(b+c+d) + b(c+d) + c(d)]\\[3ex]
\mathbf{N} &= 5\\
\Delta &= 2~[a(b+c+d+e) + b(c+d+e) + c(d+e) + d(e)]
\intertext{The last step needed to rewrite this mathematically is to notice the range of the terms. \newline Firstly, the outer terms start with a, and end at $(N-1)$, such as when $N=4$: we start with $a$, and end at $c$ (3rd term). }
\Delta &= 2~[\red{a}(...) + \red{b}(...) + \red{c}(...) ]
\intertext{Written as a sum, where we label the current term as $i$:}
&\sum_{i=1}^{N-1}i
\intertext{Secondly, the inner terms start with the next letter after the outer term ($i+1$), and end at $N$, such as when $N=4$: for the first bracket, we start with $b$ (since the outer factor is $a$) and end at $d$.}
\Delta &= 2~[a(\red{b+c+d}) + b(\red{c+d}) + ...]
\intertext{Again, written as a sum, realizing there is a bracket for each outer term (each $i$), and labellling each term as $j$:}
&\sum_{j=i+1}^{N}j
\\~\\
%
\intertext{Putting it together, recall we are iterating over the letters starting from $a$ ($1$), up until $(N-1)$, and every time multiply this by the second sum described (added terms $i+1$ to $N$). The double sum emerges:}
&\sum_{i=1}^{N-1}\sum_{j=i+1}^{N}i\cdot j
\intertext{We must of course not forget the factor of $2$ applied to this whole expression we conveniently ignored in the previous steps for simplification. The very final double sum is written as such:}
&2\sum_{i=1}^{N-1}\sum_{j=i+1}^{N}i\cdot j
\end{align*}
\begingroup
\addtolength{\jot}{1em}
\subsection*{Deriving the formula}
\newpage
\begin{flalign*}
&2\sum_{i=1}^{N-1}\sum_{j=i+1}^{N}i\cdot j & \text{derived sum} \\
&2\sum_{i=1}^{N-1}i\sum_{j=i+1}^{N}j & c\sum_{i=1}^{n}a_i=\sum_{i=1}^{n}ca_i\\
&2\sum_{i=1}^{N-1}i\left(\sum_{j=1}^{N}j-\sum_{j=1}^{i}j\right) & \text{by }\textbf{Corollary 1} \\
&2\sum_{i=1}^{N-1}i\left(\frac{N(N+1)}{2}-\frac{i(i+1)}{2}\right) & \text{by }\textbf{Lemma 2.1}\\
&2\sum_{i=1}^{N-1}\frac{i}{2}[N(N+1)-i(i+1)] & \text{factor out }\frac{1}{2}\\
&\sum_{i=1}^{N-1}i[N(N+1)-i(i+1)]& c\sum_{i=1}^{n}a_i=\sum_{i=1}^{n}ca_i\\ 
&\sum_{i=1}^{N-1}i[N^2+N-i^2-i)]&\text{distribute all brackets}\\
&\sum_{i=1}^{N-1}(iN^2+iN-i^3-i^2) & \\
&\sum_{i=1}^{N-1}iN^2+\sum_{i=1}^{N-1}iN-\sum_{i=1}^{N-1}i^3-\sum_{i=1}^{N-1}i^2 & \sum_{i=1}^{n}(a_i+b_i)=\sum_{i=1}^{n}a_i+\sum_{i=1}^{n}b_i\\
&N^2\sum_{i=1}^{N-1}i+N\sum_{i=1}^{N-1}i-\sum_{i=1}^{N-1}i^3-\sum_{i=1}^{N-1}i^2 & c\sum_{i=1}^{n}a_i=\sum_{i=1}^{n}ca_i\\
&N^2\frac{N(N-1)}{2}+N\frac{N(N-1)}{2}-\frac{N^2(N-1)^2}{4}-\frac{N(N-1)(2N-1)}{6} & \text{by }\textbf{Lemmas 2.1, 2.2, 2.3}\\
&N(N-1)\left(\frac{N^2}{2}+\frac{N}{2}-\frac{N(N-1)}{4}-\frac{2N-1}{6}\right) & \text{factor out }N(N-1)\\
&N(N-1)\left(\frac{6N^2}{12}+\frac{6N}{12}-\frac{3N^2-3N}{12}-\frac{4N-2}{12}\right) &  \text{combine fractions}\\
&N(N-1)\frac{3N^2+5N+2}{12} \\
&N(N-1)\frac{(3N+2)(N+1)}{12} & \text{factor the numerator}\\
&N\frac{(N^2-1)(3N+2))}{12} & (a+b)(a-b)=a^2-b^2\\
&N\frac{3N^3+2N^2-3N-2}{12} & \text{distribute and simplify}\\
&\frac{3N^4+2N^3-3N^2-2N}{12}\\
\end{flalign*}
\text{And we arrive at a closed form formula for the sum square difference for any $N$.}\\
\subsection*{Answering the question ($N=100$)}
\vspace{10pt}
\begin{graybox}
Find the difference between the sum of the squares of the first \textbf{one hundred} natural numbers and the square of the sum.
\end{graybox}
\vspace{15pt}
Using the formula derived earlier:
\begin{flalign*}
&\frac{3(100)^4+2(100)^3-3(100)^2-2(100)}{12}\\
&\frac{3(100000000) + 2(1000000) - 3(10000) - 2(100)}{12}\\
&\frac{300000000 + 2000000 - 30000 - 200}{12}\\
&\frac{302000000 - 30200}{12}\\
&\frac{301969800}{12}\\
&= \mathbf{25164150}
\end{flalign*}
\endgroup
\newpage
\setbool{@fleqn}{true}
\pdfminorversion=4
\documentclass[]{article}

%%%%%%%%%%%%%%%%%%%
% Packages/Macros %
%%%%%%%%%%%%%%%%%%%
\usepackage{amssymb,latexsym,geometry,xcolor,framed,url,xr}     % Standard packages
\usepackage{amsmath}
\usepackage{colors}
%%%%%%%%%%%%
% Document %
%%%%%%%%%%%%
\newgeometry{left=1cm,top=1cm,right=1cm,bottom=2cm}
\begin{document}
\allowdisplaybreaks
\raggedbottom
\section*{Euler Project - Problem 6 - Appendix A}
\subsection*{Corollaries \& Identities}
\subsubsection*{Lemma 1}
Let $j$ be any natural number between $1$ and $N$. Then,
\[\sum_{i=1}^{N}i=\sum_{i=1}^{j}i + \sum_{i=j+1}^{N}i\]\\
\textit{Proof:}\
The sum on the left may be written as:
\[1 + 2 + 3 + ... + N\]
Since $j$ is a natural number between $1$ and $N$, this is equivalent to: 
\[1 + 2 + ... + j + ... + N\]
As well as:
\[1 + 2 + ... + j + (j+1) + ... + N\]
Then, by associativity of addition in $\mathbb{R}$ (and consequently $\mathbb{N}$):
\[1 + 2 + ... + j + (j+1) + ... + N = (1 + 2 + ... + j) + ((j+1) + ... + N)\]
Which, writing the sequences as sums again, gives:
\[\sum_{i=1}^{N}i=\sum_{i=1}^{j}i + \sum_{i=j+1}^{N}i\]

\subsubsection*{Corollary 1}
Rewriting the equation to express the third sum from the left gives:
\[\sum_{i=j+1}^{N}i=\sum_{i=1}^{N}i-\sum_{i=1}^{j}i\]

\subsubsection*{Lemma 2.1}
Let $S$ be the sum of an arithmetic series of natural numbers from $1$ to $N$, where $N \in \mathbb{N}$. Then,
\[S=\sum_{i=1}^{N}i=\frac{N(N+1)}{2}\]
\textit{Proof:} Recall the formula for the sum of an arithmetic series of the form:
\[\frac{n}{2}(a_1 + a_n)\]
where $n$ is the number of terms, $a_1$ is the first term, and $a_n$ is the last term.\\
Since the number of terms is $N$, which is also the last term, since the first term is $1$, then:
\[\frac{n}{2}(a_1 + a_n) = \frac{N}{2}(1 + N) = \frac{N(N+1)}{2}\]

\subsubsection*{Lemma 2.2}
Let $S$ be the sum of an arithmetic series of squared natural numbers from $1$ to $N$, where $N \in \mathbb{N}$. Then,
\[S=\sum_{i=1}^{N}i^2=\frac{N(N+1)(2N+1)}{6}\]
\textit{Proof:} A sample proof using mathematical induction may be found at: \url{https://www.youtube.com/watch?v=JiBd5yyYfUE}
\textit{Alternate proof:} An intuitive visual proof using 3D cubes may be seen at: \url{https://www.youtube.com/watch?v=aXbT37IlyZQ}
\subsubsection*{Lemma 2.3}
Let $S$ be the sum of an arithmetic series of cubed natural numbers from $1$ to $N$, where $N \in \mathbb{N}$. Then,
\[S=\sum_{i=1}^{N}i^3=\left(\sum_{i=1}^{N}i\right)^2=\left(\frac{N(N+1)}{2}\right)^2=\frac{N^2(N+1)^2}{4}\]
\textit{Proof:} A sample proof using mathematical induction may be found at: \url{https://www.youtube.com/watch?v=w362XRZy5as}
\textit{Alternate proof:} An intuitive visual proof once again using 3D cubes may be seen at: \url{https://www.youtube.com/watch?v=w362XRZy5as}
\end{document}

\end{document}
